\documentclass[11pt]{article}
\usepackage[margin=1in]{geometry}
\usepackage{amsmath}
\usepackage{amsfonts}
\usepackage{amssymb}
\usepackage{listings}
\usepackage{xcolor}
\usepackage{fancyhdr}
\usepackage{titlesec}
\usepackage{enumitem}

% Code listing settings
\lstset{
    basicstyle=\footnotesize\ttfamily,
    breaklines=true,
    frame=single,
    numbers=left,
    numberstyle=\tiny,
    keywordstyle=\color{blue},
    commentstyle=\color{green!60!black},
    stringstyle=\color{red},
    backgroundcolor=\color{gray!10},
    showspaces=false,
    showstringspaces=false,
    tabsize=2
}

% Page setup
\pagestyle{fancy}
\fancyhf{}
\rhead{LeetCode Notes}
\lhead{Difficult Problems}
\rfoot{\thepage}

% Section formatting
\titleformat{\section}{\Large\bfseries}{\thesection}{1em}{}
\titleformat{\subsection}{\large\bfseries}{\thesubsection}{1em}{}

\begin{document}

\title{\textbf{LeetCode Difficult Problems Notes}}
\author{}
\date{}
\maketitle

\tableofcontents
\newpage

% Section for difficult LeetCode problems

\section{Tree Problems}

\subsection{110. Balanced Binary Tree}

\textbf{Problem:} Given a binary tree, determine if it is height-balanced. A height-balanced binary tree is defined as a binary tree in which the height of the left and right subtree of every node differs in height by no more than 1.

\textbf{Approach:} Use a bottom-up approach with a helper function that returns the height of the subtree. If any subtree is unbalanced, propagate -1 upwards.

{\small
\begin{lstlisting}[language=C++]
/**
 * Definition for a binary tree node.
 * struct TreeNode {
 *     int val;
 *     TreeNode *left;
 *     TreeNode *right;
 *     TreeNode() : val(0), left(nullptr), right(nullptr) {}
 *     TreeNode(int x) : val(x), left(nullptr), right(nullptr) {}
 *     TreeNode(int x, TreeNode *left, TreeNode *right) : val(x), left(left), right(right) {}
 * };
 */

// This class checks if a given binary tree is height-balanced.
class Solution {
public:
    /**
     * The main function to check if a tree is balanced.
     * A balanced tree is defined as one where the heights
     * of the two child subtrees of any node never differ by more than one.
     */
    bool isBalanced(TreeNode* root) {
        // If checkHeight returns -1, the tree is not balanced.
        return checkHeight(root) != -1;
    }

    /**
     * Helper function to compute the height of the tree.
     * Returns -1 immediately if an unbalanced subtree is found.
     */
    int checkHeight(TreeNode* root) {
        // An empty subtree is balanced, and its height is 0.
        if (root == nullptr) {
            return 0;
        }
        
        // Recursively check the height of the left subtree.
        int leftHeight = checkHeight(root->left);
        // If left subtree is unbalanced, propagate failure upwards (-1).
        if (leftHeight == -1) {
            return -1;      
        }

        // Recursively check the height of the right subtree.
        int rightHeight = checkHeight(root->right);
        // If right subtree is unbalanced, propagate failure upwards (-1).
        if (rightHeight == -1) {
            return -1;
        }

        // If the current node is unbalanced (heights differ by more than 1), report unbalanced.
        if (abs(leftHeight - rightHeight) > 1) {
            return -1;
        }

        // If balanced, return height of current subtree.
        // Height is 1 + maximum height of the two subtrees.
        return 1 + max(leftHeight, rightHeight);
    }
};

/*
  Explanation of the -1 "signal":
  --------------------------------
  The function uses -1 as a special value to indicate that 
  an unbalanced subtree has been found. Once -1 is returned 
  by any recursion, all further ancestors also return -1,
  which efficiently stops further unnecessary checks.
*/
\end{lstlisting}
}

\textbf{Time Complexity:} O(n) - we visit each node once\\
\textbf{Space Complexity:} O(h) - where h is the height of the tree (worst case O(n) for skewed trees)

\newpage


\end{document}
